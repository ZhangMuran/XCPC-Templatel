\ifx\allfiles\undefined
\documentclass[a4paper]{book}
\usepackage{ctex}
\usepackage{amsmath,amsthm}
\usepackage{lmodern}
\usepackage{listings,xcolor} %代码块
\usepackage{xcolor}
\usepackage{listings}
\lstset{
    breaklines,                                 % 自动将长的代码行换行排版
    extendedchars=false,                        % 解决代码跨页时,章节标题,页眉等汉字不显示的问题
    backgroundcolor=\color[rgb]{0.96,0.96,0.96},% 背景颜色
    keywordstyle=\color{blue}\bfseries,         % 关键字颜色
    identifierstyle=\color{black},              % 普通标识符颜色
    commentstyle=\color[rgb]{0,0.6,0},          % 注释颜色
    stringstyle=\color[rgb]{0.58,0,0.82},       % 字符串颜色
    showstringspaces=false,                     % 不显示字符串内的空格
    numbers=left,                               % 显示行号
    numberstyle=\small\ttfamily,                % 设置数字字体
    basicstyle=\small\ttfamily,                 % 设置基本字体
    captionpos=t,                               % title在上方(在bottom即为b)
    frame=single,                               % 设置代码框形式
    rulecolor=\color[rgb]{0.8,0.8,0.8},         % 设置代码框颜色
}  
   

\begin{document}
\fi

\section{背包DP}
\subsection*{多重背包}
多重背包也是0-1背包的一个变式。与0-1背包的区别在于每种物品有$k_i$个,而非一个。一个很朴素的想法就是:把「每种物品选$k_i$次」等价转换为「有$k_i$个相同的物品,每个物品选一次」。这样就转换成了一个0-1背包模型,套用上文所述的方法就可已解决。状态转移方程如下:
\begin{lstlisting}[language=c++]
for(int i=0;i<n;i++)
{
    for(int j=v;j>=0;j--)
    {
        for(int k=0;k<=s[i]&&k*w[i]<=j;k++)
        dp[j]=max(dp[j],dp[j-k*w[i]]+k*val[i]);
    }
}
\end{lstlisting}
\subsubsection{二进制分组优化}
\indent对于一个数量为$M_i$件的物品,将其分成若干见,令这些系数分别为$1,2,2^2...2^{k-1},M_i-2^k+1$,继而转化为01背包,就使得本为$O(V\sum M_i)$时间复杂度的问题变为了$O(V\sum logM_i)$。
\subsubsection{单调队列优化}
\indent对于价值为$v$,重量为$w$,数量为$k$的物品,$dp[m]$的状态只与$dp[m-w]+v,dp[m-2*w]+2*v,\cdots,dp[m-k*v]+k*v$有关。所以我们将这个问题等价与$w$个同模的单调队列,在其上进行长度为$k$的滑动窗口问题。每次加入队列的值为$dp[j+k*v]-k*w$。时间复杂度$O(NV)$。
$$
\begin{aligned}
&dp[j]=dp[j]\\
&dp[j+v]  = max(dp[j], dp[j+v] - w) + w\\
&dp[j+2v] = max(dp[j], dp[j+v] - w, dp[j+2v] - 2w) + 2w\\
&dp[j+3v] = max(dp[j], dp[j+v] - w, dp[j+2v] - 2w, dp[j+3v] - 3w) + 3w\\
\end{aligned}
$$
\begin{lstlisting}[language=c++]
for(int i=1;i<=N;i++)
{
    scanf("%d%d%d",&v,&w,&s);
    int u=i%2;
    for(int j=0;j<=V;j++) dp[u][j]=dp[u^1][j];
    for(int j=0;j<v;j++)
    {
        int head=0,tail=-1;
        for(int k=j;k<=V;k+=v)
        {
            if(head<=tail && k-s*v>q[head]) ++head;
            while(head<=tail&&dp[u^1][q[tail]]-q[tail]/v*w<=dp[u^1][k]-k/v*w) 
            --tail;
            if(head<=tail)
            dp[u][k]=max(dp[u][k],dp[u^1][q[head]]+(k-q[head])/v*w);
            q[++tail]=k;
        }
    }
}
printf("%d\n",dp[N%2][V]);
\end{lstlisting}
\subsection*{混合背包}
\indent混合背包就是将前面三种的背包问题混合起来,有的只能取一次,有的能取无限次,有的只能取$k$次。
\begin{lstlisting}
for(循环物品种类) 
{
    if(是0-1背包) 套用0-1背包代码;
    else if(是完全背包) 套用完全背包代码;
    else if(是多重背包) 套用多重背包代码;
}
\end{lstlisting}
\subsection*{二维费用背包}
\indent有$n$个任务需要完成,完成第$i$个任务需要花费$t_i$分钟,产生$c_i$元的开支。现在有$T$分钟时间,$W$元钱来处理这些任务,求最多能完成多少任务。\\
\indent选一个物品会消耗两种价值(经费、时间),只需在状态中增加一维存放第二种价值即可。
\begin{lstlisting}[language=c++]
for(int k=1;k<=n;k++) 
{
    for(int i=m;i>=mi;i--)    // 对经费进行一层枚举
        for(int j=t;j>=ti;j--)  // 对时间进行一层枚举
            dp[i][j]=max(dp[i][j],dp[i-mi][j-ti]+1);
}
\end{lstlisting}
\subsection*{分组背包}
\indent有$n$件物品和一个大小为$m$的背包,第$i$个物品的价值为$v_i$,体积为$w_i$。同时,每个物品属于一个组,同组内最多只能选择一个物品。求背包能装载物品的最大总价值。\\
\indent这种题怎么想呢?其实是从「在所有物品中选择一件」变成了「从当前组中选择一件」,于是就对每一组进行一次0-1背包就可以了。再说一说如何进行存储。我们可以将$t_{k,i}$表示第$k$组的第$i$件物品的编号是多少,再用$cnt_k$表示第$k$组物品有多少个。
\begin{lstlisting}[language=c++,escapeinside=``]
for(int k=1;k<=ts;k++) //循环每一组
    for(int i=m;i>=0;i--) //循环背包容量
        for(int j=1;j<=cnt[k];j++) //循环该组的每一个物品
        if(i>=w[t[k][j]])
        dp[i]=max(dp[i],dp[i-w[t[k][j]]]+c[t[k][j]]); //`像0-1背包一样状态转移`
\end{lstlisting}
\subsection*{有依赖的背包}
对于一个主件和它的若干附件,有以下几种可能:只买主件,买主件$+$某些附件。因为这几种可能性只能选一种,所以可以将这看成分组背包。如果是多叉树的集合,则要先算子节点的集合,最后算父节点的集合。
\subsection*{泛化物品}
这种背包,没有固定的费用和价值,它的价值是随着分配给它的费用而定。在背包容量为$V$的背包问题中,当分配给它的费用为$v_i$时,能得到的价值就是$h(v_i)$。这时,将固定的价值换成函数的引用即可。
\subsection*{杂项}
\subsubsection{输出方案}
\begin{lstlisting}[language=c++]
for(int i=N;i>=1;i--)
{
    for(int j=0;j<=V;j++)
    {
        f[i][j]=f[i+1][j];
        if(j>=v[i]) f[i][j]=max(f[i][j],f[i+1][j-v[i]]+w[i]);
    }
}
int j=V;
for(int i=1;i<=N;i++)
{
    if(j>=v[i] && f[i][j]==f[i+1][j-v[i]]+w[i])
    {
        j-=v[i];printf("%d ",i);
    }
}
\end{lstlisting}
\subsubsection{求方案数}
对于给定的一个背包容量、物品费用、其他关系等的问题,求\textbf{装到一定容量}的方案总数。这种问题就是把求最大值换成求和即可。例如0-1背包问题的转移方程就变成了:
$$
\mathit{dp}_i=\sum(dp_i,dp_{i-c_i})
$$
初始条件:$\mathit{dp}_0=1$,因为当容量为$0$时也有一个方案,即什么都不装。
\subsubsection{求最优方案总数}
\indent要求最优方案总数,我们要对0-1背包里的$dp$数组的定义稍作修改,DP状态$f_{i,j}$为在只能放前$i$个物品的情况下,容量为$j$的背包「正好装满」所能达到的最大总价值。\\
\indent这样修改之后,每一种DP状态都可以用一个$g_{i,j}$来表示方案数。\\
\indent$f_{i,j}$表示只考虑前$i$个物品时背包体积「正好」是$j$时的最大价值。\\
\indent$g_{i,j}$表示只考虑前$i$个物品时背包体积「正好」是$j$时的方案数。\\
\indent如果$f_{i,j}=f_{i-1,j}$且$f_{i,j}\neq f_{i-1,j-v}+w$说明我们此时不选择把物品放入背包更优,方案数由$g_{i-1,j}$转移过来,\\
\indent如果$f_{i,j}\neq f_{i-1,j}$且$f_{i,j}=f_{i-1,j-v}+w$说明我们此时选择把物品放入背包更优,方案数由$g_{i-1,j-v}$转移过来,\\
\indent如果$f_{i,j}=f_{i-1,j}$且$f_{i,j}=f_{i-1,j-v}+w$说明放入或不放入都能取得最优解,方案数由$g_{i-1,j}$和$g_{i-1,j-v}$转移过来。
\begin{lstlisting}[language=c++]
for(int i=1;i<=n;i++)
{
    int v,w;cin>>v>>w;
    for(int j=m;j>=v;j--)
    {
        int maxx=max(f[j],f[j-v]+w),cnt=0;
        if(maxx==f[j]) cnt+=g[j];
        if(maxx==f[j-v]+w) cnt+=g[j-v];
        g[j]=cnt%mod;f[j]=maxx;
    }
}
\end{lstlisting}
\subsubsection{第$k$优解}
$dp_{i,j,k}$记录了前$i$个物品中,选择的物品总体积为$j$时,能够得到的第$k$大的价值和。转移时,与仍然采用背包原来的转移方式。不同的是现在需要记录的是所有可能情况的一个序列,使用双指针维护即可。
\begin{lstlisting}[language=c++]
for(int i=1;i<=n;i++)
{
    for(int j=V;j>=w[i];j--)
    {
        for(int p=1;p<=k;p++)
        {
            a[p]=f[j-w[i]][p]+v[i];b[p]=f[j][p];
        }
        int x=1,y=1,z=1;a[k+1]=b[k+1]=-1;
        while(z<=k&&(a[x]!=-1||b[y]!=-1)) 
        {
            if(a[x]>b[y]) f[j][z]=a[x++];
            else f[j][z]=b[y++];
            if(f[j][z]!=f[j][z-1]) z++;
        }
    }
}
printf("%d\n",f[V][k]);
\end{lstlisting}

\section{树形DP}
\subsection*{树上背包}
\indent现在有$n$门课程,第$i$门课程的学分为$a_i$,每门课程有零门或一门先修课,有先修课的课程需要先学完其先修课,才能学习该课程。一位学生要学习$m$门课程,求其能获得的最多学分数。\\
\indent我们可以新增一门$0$学分的课程(设这个课程的编号为$0$),作为所有无先修课课程的先修课,这样我们就将森林变成了一棵以$0$号课程为根的树。\\
\indent使用上下界优化后的时间复杂度为$O(nm)$。
\begin{lstlisting}[language=c++,escapeinside=``]
int dfs(int u)
{
    int sz=1;
    f[u][1]=w[u];
    for(int i=head[u];~i;i=r[i].nex)
    {
        int v=r[i].b;
        int cnt=dfs(v);
        for(int j=min(sz,m+1);j>=1;j--) //上下界优化
        { //`m+1`是因为加上新建的根结点一共选了`m+1`门课
            for(int k=1;k<=cnt&&j+k<=m+1;k++) //`这里k从1开始,如果需要从0则最后处理`
                f[u][j+k]=max(f[u][j+k],f[u][j]+f[v][k]);
        }
        sz+=cnt;
    }
    return sz;
}
\end{lstlisting}
\section{数位DP}
\indent对于多组数据且状态数较多的数位DP,如何保证可以重复使用DP数组且不清空就成为了提升效率的关键。\\
\indent发现有一个东西在阻止我们复用DP数组,那就是limit。对于不同的输入,limit的意义本质上是不同的。不过我们注意到limit等于$1$的状态出现的频率远远小于limit等于$0$的状态,所以我们可以选择只记忆化limit为$0$的状态,这样每次DP数组的意义就完全相同了。
\begin{lstlisting}[language=c++]
ll dfs(int pos,int lead, int limit)
{
    ll ans = 0;
    if (!pos) return ...;
    ll &d = dp[pos][...];
    if (!limit && d != -1) return d;
    for (int v = 0;v<=(limit ? A[pos] : 9); ++v)
        ans += dfs(pos - 1, lead&&v==0, limit && A[pos] == v);
    if (!limit) d = ans;
    return ans;
}
ll f(ll x)
{
    int len = 0;
    while (x) A[++len] = x % 10, x /= 10;
    return dfs(len, ..., true);
}
\end{lstlisting}
\section{概率DP}
\indent一个软件有$s$个子系统,会产生$n$种bug。某人一天发现一个bug,这个bug属于某种bug分类,也属于某个子系统。每个bug属于某个子系统的概率是$\frac{1}{s}$,属于某种bug分类的概率是$\frac{1}{n}$。求发现$n$种bug,且$s$个子系统都找到bug的期望天数。\\
\indent令$f_{i,j}$为已经找到$i$种bug分类,$j$个子系统的bug,达到目标状态的期望天数。考虑$f_{i,j}$的状态转移:
\begin{itemize}
    \item $f_{i,j}$,发现一个bug属于已经发现的$i$种bug分类,$j$个子系统,概率为$p_1=\frac{i}{n}\cdot\frac{j}{s}$。
    \item $f_{i,j+1}$,发现一个bug属于已经发现的$i$种bug分类,不属于已经发现的子系统,概率为$p_2=\frac{i}{n}\cdot(1-\frac{j}{s})$。
    \item $f_{i+1,j}$,发现一个bug不属于已经发现bug分类,属于$j$个子系统,概率为$p_3=(1-\frac{i}{n})\cdot\frac{j}{s}$。
    \item $f_{i+1,j+1}$,发现一个bug不属于已经发现bug分类,不属于已经发现的子系统,概率为$p_4=(1-\frac{i}{n})\cdot(1-\frac{j}{s})$。  
\end{itemize}
\indent再根据期望的线性性质,就可以得到状态转移方程:
$$
\begin{aligned}
    f_{i,j}&=p_1\cdot f_{i,j}+p_2\cdot f_{i,j+1}+p_3\cdot f_{i+1,j}+p_4\cdot f_{i+1,j+1}+1\\
           &=\frac{p_2\cdot f_{i,j+1}+p_3\cdot f_{i+1,j}+p_4\cdot f_{i+1,j+1}+1}{1-p_1}
\end{aligned}
$$
\ifx\allfiles\undefined
\end{document}
\fi
